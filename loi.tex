\documentclass{article}
\usepackage[colorlinks = true,
linkcolor = blue,
urlcolor  = blue,
citecolor = blue,
anchorcolor = blue]{hyperref}

\usepackage{authblk}
\title{Diversity and Inclusion: Making Physics Accessible}
\author[a]{Giordon Stark}
\author[b]{Amber Roepe}
%\author[a]{Wanderer}
%\author[a]{Static}
\affil[a]{SCIPP, UC Santa Cruz}
\affil[b]{University of Oklahoma}
\date{}
%\setcounter{Maxaffil}{0}
\renewcommand\Affilfont{\itshape\small}

\usepackage{geometry}
\geometry{letterpaper, margin=1in}

\usepackage[backend=biber,sorting=none,backref=true]{biblatex}
\addbibresource{cites.bib}

\usepackage{enumitem}

\begin{document}
\maketitle
\hrule
\vspace{0.5em}
\hrule
\vspace{2.5em}
% begin main text ~

  Diversity and Inclusion (D\&I) plays a very important role in society at large. Physics, in general, has been notoriously slow in improving its representation and accessibility - however this is certainly gaining more attention in recent times. Research shows that diversity works to make us smarter~\cite{HowDiversityWorks} and that socially diverse groups (i.e. diversity of race, ethnicity, gender and/or sexual orientation) are more innovative than homogeneous groups. By including and interacting with diverse individuals, the group quickly learns to be better prepared, to anticipate needs, to understand alternative viewpoints, and to put effort into reaching a consensus.

  There are many ways to promote this effort within any organization, from encouraging younger members to ask questions first after a presentation to ensuring equitable access to restrooms and quiet areas at a conference. This letter of interest is aimed to suggest a whitepaper that will document (perhaps in a checklist form) ways to improve and make Science more inclusive, with a light shown introspectively on the Snowmass 2021 process in regards to accessibility. This allows the whitepaper to be a formal part of the Snowmass 2021 process and hopefully lead to impactful dialogue at the agency level, and elsewhere.

  What is the purpose of the whitepaper? It will try to formalize all the thoughts and musings that we've tried to organize in this Letter of Interest. The focus should be to provide recommendations about how to be more accessible, provide resources/explanations of why closed captions can be preferable to ASL or vice-versa, and to help people understand what it means to think about making physics meetings accessible to people who are d/Deaf and Hard-of-Hearing. There are important distinctions being made between deaf vs Deaf and the reader is encouraged to review Chua, Smith, et al.~\cite{asee_peer_32676} to understand the sociolinguistic and cultural differences between the two groups.

  Some of the important and necessary questions that will be answered by the whitepaper are as follows:


\begin{itemize}
  \item Who is responsible for providing access?
  \item Who is responsible for paying for access?
  \item What kind of access to provide?
  \item Are meeting transcripts sufficient?
  \item Is auto-captioning useful/sufficient?
  \item What kind of additional materials should be provided?
\end{itemize}


  These are the types of questions that should be asked by organizers, by conveners, and by meeting hosts, and our goal is to provide a succinct document addressing these questions for future proceedings. The US ATLAS Diversity and Inclusion Committee has already composed, and implemented, a checklist~\cite{USATLASChecklist}  with D\&I in mind. This checklist was developed as a guide for all hosts of the US ATLAS annual meeting every year to follow to ensure equitable, diverse, and inclusive meetings. With this document as a base, we would like to develop a similar checklist for future Snowmass and DPF meetings to ensure the environment is diverse and inclusive.

 \printbibliography
\end{document}
  
  
  
